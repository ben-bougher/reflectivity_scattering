\documentclass{article} % For LaTeX2e
\usepackage{nips14submit_e,times}
\usepackage{hyperref}
\usepackage{url}
\usepackage{mathtools}
\usepackage{float}
%\documentstyle[nips14submit_09,times,art10]{article} % For LaTeX 2.09


\title{Predicting stratigraphic units from multifractal properties of
  well logs}


\author{
Ben Bougher\\
Department of Earth, Ocean, and Atmospheric Science\\
University of British Columbia\\
Vancouver, BC\\
\texttt{ben.bougher@gmail.com} 
}

% The \author macro works with any number of authors. There are two commands
% used to separate the names and addresses of multiple authors: \And and \AND.
%
% Using \And between authors leaves it to \LaTeX{} to determine where to break
% the lines. Using \AND forces a linebreak at that point. So, if \LaTeX{}
% puts 3 of 4 authors names on the first line, and the last on the second
% line, try using \AND instead of \And before the third author name.

\newcommand{\fix}{\marginpar{FIX}}
\newcommand{\new}{\marginpar{NEW}}

\nipsfinalcopy % Uncomment for camera-ready version

\begin{document}


\maketitle

\begin{abstract}
The sedimentary layers of the Earth are built up by random depositional
processes over different timescales. These processes form multifractal
relationships in the strata of the Earth, which is observed in the
geophysical signals we measure. Making use of the scattering
transform, we aim to exploit these multifractal relationships in
order to predict stratigraphic units from geophysical measurements. We
try our approach using labelled well log data from the Trenton Black
River Project. We train a classifier using scattering transform
coefficients as features, and assess the ability to predict
stratigraphic units from gamma-ray well logs. The scattering based
classifier predicted 5 different stratigraphic units with a success
rate of .65,which significantly outperformed another wavelet-based
approach (.40).
\end{abstract}

\section{Introduction}


\section{Background}
Structures in nature exhibit fractal geometry, where a complex
irregular structure reoccurs at every scale[1]. This fractal nature of the
Earth's stratigraphic layers is contained in the well log
data. Analysis of fractals by Mandelbrot[1] demonstrated that seemingly complex geometries could
be represented by a very simple recursive relationship. This
formulation has been exploited by the special effects industry, where
realistic landscapes are synthesized with a simple
recursive parameterization. A desert or a mountain landscape
can be rendered based on differing fractal relationships, an insight
that motivates the use of fractals for classifying
geological groups. 

Machine learning from well log data is an active field of research,
yet remains relatively sparse in scientific
literature. Work by Salehi and Honarvar[2] demonstrated success predicting 
photoelectric adsorption from other measurements, but this type of numerical prediction is not
as novel as predicting a geologists interpretation. This project instead
follows an approach similar to [3] where Holder exponents
of wavelets were used as inputs into an artificial neural network. Holder
exponents are used to analyze fractal properties of signals[4], which
make them an intuitive discriminating feature. The approach was ineffective at
predicting thin bed lithologies, as the wavelet transform removed
much of the high-resolution geological features. Wavelet transforms
may be better suited for stratigraphic analysis, where thin beds
are grouped into larger geological units.

Higher-order wavelet transforms, such as the scattering transform[5],
use a neural network of cascading wavelet transforms to characterize
signals. The spectral co-occurance coefficients (SCOC) output by the
scattering transform contain multi-scale information about the signal,
making them a promising basis for characterizing fractal relationships.
The scattering transform has demonstrated success in classifying audio signals
into genres of music[6], which is a problem of predicting human
interpretation of a 1D signal. The fractal nature of the scattering
transform and its prior success with classifying human interpreted data
makes SCOCs a promising feature basis for well log learning.


\section{Methodology}
A supervised learning approach was applied to gamma-ray measurements
with labelled stratigraphic units. A scattering transform was used to
extract SCOCs as features from gamma-ray measurements, which were 
input into a KNN classifier. 

\subsection{Dataset}
In the early 2000s, the Trenton Black River carbonates experienced renewed interest due to
speculation among geologists of potential producing reservoirs. A basin
wide collaboration to study the region resulted in many data products,
including well logs with corresponding stratigraphic analysis. The
dataset  contained 80 gamma-ray logs with
corresponding stratigraphic labels, and an additional 70 unlabelled
logs. Table \ref{strat-table} contains the number of feature samples
for each stratigraphic unit and Figure ~\ref{fig:log} shows an example
of a labelled log. The dataset can be downloaded at [7].

\subsection{Feature extraction}
Well logs are not uniform measurements, as each log will have
different start and stop depths and contain multiple stratigraphic
labels. This type of data does not fit  directly into a classification scheme, so an extraction
to a new feature basis was required. The scattering transform, shown
in Figure ~\ref{fig:scatter}, was used to transform well logs into
spectral co-occurance coefficients. The scattering transform takes on the
structure of a trained neural network where each
stage is a wavelet transform. The resulting SCOCs are multi-scale
measurements of signals, making them well-suited for the fractal
structure of well logs. 

A full description of the scattering transform is contained in [5],
so this paper will only provide a quick overview. Let $x(t)$ be time series
data, $\phi_{J}(t)$ be a low-pass filter and $\psi_{j}(t)$ be an
arbitrary wavelet transform. The scattering transform
\begin{equation}
S_{\phi}x(t) =
\begin{pmatrix}
x \star \phi(t)\\
|x \star \psi_{\lambda_1}|\star \phi(t) \\
\|x \star \psi_{\lambda_1}| \star \psi_{\lambda_2}| \star \phi(t) \\
\vdots \\
\| \hdots |x \star \psi_{\lambda_1} \hdots | \star \psi_{\lambda_n}| \star \phi(t)
\end{pmatrix}
\end{equation}
is a cascade of modulated filter banks and non-linear
rectifications. Each row $n$ of the vector contains the time dependent
SCOCs for the $n^{th}$ layer of the network. 

An audio wavelet bank with 128 sample averaging was chosen for
the wavelet transform basis, and a network depth of 2 layers was used
for extracting SCOCs as feature vectors. The feature data contained
contained 50 SCOCs for each averaged time sample.

The MATLAB code $ScatNet$[8] was used for
calculating SCOCs.

\subsection{Classification}
The labelled dataset was split into training and testing datasets. A
classification was made by calculating the euclidean distance between
each testing and training vector and choosing the mode of the k
closest neighbours. Cross-validation was used to determine a
value of 14 for $k$.

\section{Results}
Table ~\ref{results-table} shows the classification accuracy for each
stratigraphic group. The total classification accuracy was 36\% with
the highest accuracy(67\%) occurring in the Black River
group. Referring to Table ~\ref{strat-table}, Table ~\ref{results-table}, and Figure
~\ref{fig:log} there is direct correlation between the relative size
of the stratigraphic unit and the classification accuracy. This could
indicate that the scattering transform is only useful for characterizing
larger groups, but could also be a statistical artefact related sample
sizes. 

\section{Discussion and future work}
This was a novel approach to applying machine learning to a relatively
new domain. These initial results showed promise, especially for
larger stratigraphic units. This paper applied the method to uncleaned real world data and yielded
some positive results.The parameter space in a machine learning algorithm is massive, and this project
explored a tiny portion of it. The scattering transform offers many
degrees of freedom; wavelet bank selection, sampling window, and
network depth all have a significant impact on the features that get
extracted. Future work will configure the scattering transform for well log data.

KNN classification is often successful, but requires an
accurate measure of feature distance. Euclidean distance is likely an ill-suited measure
of similarity between SCOC vectors, as large differences between a few
coefficients will equate to large distance. I also infer that a KNN
classifier would be sensitive to classes with few members. A feature selection stage,
or a PCA on the feature vectors could improve the classification
results. A reduction in the dimensionality of the features would allow
for an insightful  visual analysis. Smaller feature dimensions would work better for probabilistic discrimination
methods such as Gaussian discriminant analysis. Since the number of
stratigraphic labels is known, this problem would be well-suited
for a Gaussian mixture model.

I assumed no prior knowledge of how strata are formed and
interact, knowledge that is well-known in the geological
community. For example, the law of superposition states: \textit{``Sedimentary
layers are deposited in a time sequence, with the oldest on the bottom
and the youngest on the top"}. This law could be applied in a Bayesian
framework to prevent predicted classifications appearing in a
non-sensical orientations. Additional knowledge of common geologic
structures could be applied to regularize the predictions. Applying
\textit{a priori} geological information would allow for feasibility
constraints that could improve performance.

Assessment of the success of the method does not fit into a typical
binary right/wrong performance metric. The method is attempting to
predict a subjective human interpretation, so a more subjective
interpretation of performance should be used. Allowing geologists to visually verify the results
would be a beneficial metric.

The Black River dataset was the only easily accessible dataset suited to this
type experiment. The data was raw, and the stratigraphic analysis was
contained in images that needed to be translated and connected to
numerical data. Scouring the internet for datasets, and manually
preparing the data is a painstaking process that severely limits the
productivity of the research. Although there is an abundance of
available well logs, corresponding stratigraphic analysis is held as
proprietary information by exploration companies. A database of open
data has benefitted many other industries, and opening a database of
labelled well logs would likely see a large increase in the
amount applied research in the domain.

\section{Conclusions}

A supervised machine learning approach was applied to stratigraphic
unit prediction from well log data. The results show initial promise, and
the motivation and method opens the door for future research.

This paper demonstrated a novel approach to a high impact problem in
a field seldom explored in machine learning literature. The paper
suggested a connection between fractal patterns observed in
nature and a neural network based multi-scale wavelet transform. The
method was demonstrated on geological field data with disparate human
interpretations. Initial results from a simple classifier show promise
for future research.

Geological analysis is often about visual interpretations
of patterns across huge multidimensional datasets, and there are
decades of interpreted datasets contained in the walls of oil
companies. The impact that machine learning can make in the exploration industry
and earth science is enormous. 
\newpage

\section{Figures}

\begin{figure}[H]
\begin{center}
\includegraphics[scale=0.5]{example_log.png}
\end{center}
\caption{Well log image with stratigraphic labels.}
\label{fig:log}
\end{figure}


\begin{figure}[H]
\begin{center}
\includegraphics[scale=0.2]{scattering_transform.png}
\end{center}
\caption{Diagram of scattering transform, taken from [5].}
\label{fig:scatter}
\end{figure}

\section{Tables}

\begin{table}[H]
\caption{Stratigraphic units}
\label{strat-table}
\begin{center}
\begin{tabular}{ll}
\multicolumn{1}{c}{\bf Unit}  &\multicolumn{1}{c}{\bf Feature Count}
\\ \hline \\
Ordovician    &671 \\
Kope             &369\\                              
Utica   &241\\
Point Pleasant  &119 \\
T Lexington   &165\\
Black River  &567 \\
Gull River  &101\\ 
Wells Creek   &57\\      
\end{tabular}
\end{center}
\end{table}

\begin{table}[H]
\caption{Classification accuracy by stratigraphic group}
\label{results-table}
\begin{center}
\begin{tabular}{ll}
\multicolumn{1}{c}{\bf Unit}  &\multicolumn{1}{c}{\bf Classification accuracy}
\\ \hline \\
Ordovician    &0.5 \\
Kope             &0.2\\                              
Utica   &0.12\\
Point Pleasant  &0.03 \\
T Lexington   &0.04\\
Black River  &0.67 \\
Gull River  &0.0\\ 
Wells Creek   &0.0\\      
\end{tabular}
\end{center}
\end{table}


\subsubsection*{Acknowledgments}

I would like to acknowledge the Black River Project for making
labelled well data publicly available, my supervisor Dr. Felix
Herrmann for sharing his knowledge of wavelets, Dr. Mark Schmidt
for teaching me everything I know about machine learning.

\subsubsection*{References}

\small{
[1] Mandelbrot, B.B. (1982) The fractal geometry of
nature. Ed. W. H. Freeman, San Francisco.

[2]Salehi, S.M. \& Honarvar (2014) Automatic Identification of
Formation lithology from Well Log Data: A machine Learning Approach. 
{\it Journal of Petroleum Science Research}
{\bf 3}(2):73-82.

[3]Leila Aliouane, Sid-Ali Ouadfeul and Amar Boudella (2012). Well-Logs Data Processing Using the Fractal Analysis and Neural Network, Fractal Analysis and Chaos in Geosciences, Dr. Sid-Ali Ouadfeul (Ed.), ISBN: 978-953-51-0729-3, InTech, DOI: 10.5772/51875. Available from: http://www.intechopen.com/books/fractal-analysis-and-chaos-in-geosciences/well-logs-data-processing-using-the-fractal-analysis-and-neural-network

[4]Herrmann, F.J. (1997)  A scaling medium representation, a
discussion on well-logs, fractals and waves, Phd thesis Delft
University of Technology, Delft, The Netherlands

[5]Deep Scattering Spectrum, Andén J. and Mallat. S., Submitted to IEEE Transactions on Signal Processing, 2011. ]

[6]Salehi, S.M. \& Honarvar (2014) Automatic Identification of
Formation lithology from Well Log Data: A machine Learning Approach. 
{\it Journal of Petroleum Science Research}
{\bf 3}(2):73-82.

[7]Trenton Black River Project (2006),
http://www.wvgs.wvnet.edu/www/tbr/

[8] ScatNet MATLAB package,
http://www.di.ens.fr/data/software/scatnet/
}

\end{document}
